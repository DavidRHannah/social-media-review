\documentclass[11pt]{article}
\usepackage[margin=1in]{geometry}
\usepackage{hyperref}
\usepackage{enumitem}
\usepackage{titlesec}
\usepackage{graphicx}
\usepackage{fancyhdr}
\usepackage{parskip}
\usepackage{sectsty}

\allsectionsfont{\sffamily}
\pagestyle{fancy}
\fancyhf{}
\lhead{MediaReview Social: Tools and Frameworks}
\rhead{\today}
\cfoot{\thepage}

\title{Title for Media Review}
\author{Your Company Name}
\date{\today}

\begin{document}
\maketitle
\tableofcontents
\newpage

\section{Heuristic-Based Recommendation

System}\label{heuristic-based-recommendation-system}



\textbf{1. Architecture \& Modularity}\\

• \textbf{Modular Design:} Separate the recommendation engine from data

ingestion, business logic, and presentation layers. This helps in

maintaining and extending the system later (for instance, when

integrating machine learning components).\\

• \textbf{Configurable Rules:} Implement heuristic rules in a

configuration file or a dedicated module. This way, updating the rules

doesn't require redeploying the entire system, and it paves the way for

a smoother transition when incorporating ML-based recommendations.



\textbf{2. Heuristic Rules \& Scoring}\\

• \textbf{Rule Definition:} Identify key factors (e.g., user behavior,

item similarity, popularity) and codify them into rules. For example,

``if a user has viewed similar items multiple times, boost the score for

those items.''\\

• \textbf{Weighted Scoring:} Assign weights to each rule to produce a

composite recommendation score. This scoring mechanism allows for

fine-tuning and easier debugging of the system's outputs.



\textbf{3. Data Considerations}\\

• \textbf{Data Preparation:} Establish a data pipeline that cleans and

aggregates user interactions and item metadata. A well-defined data

model will support both heuristic and future ML approaches.\\

• \textbf{Performance Optimization:} Consider caching frequently

computed scores or recommendations to improve responsiveness.



\begin{center}\rule{0.5\linewidth}{0.5pt}\end{center}



\section{Testing Strategy}\label{testing-strategy}



\textbf{1. Unit Testing}\\

• \textbf{Individual Rule Tests:} Create unit tests for each heuristic

rule to ensure they behave as expected under various conditions.\\

• \textbf{Edge Cases:} Test for edge cases (e.g., missing data, extreme

values) to validate the robustness of each rule.



\textbf{2. Integration Testing}\\

• \textbf{Component Interaction:} Validate the interaction between data

ingestion, the heuristic engine, and output formatting. This ensures

that data flows correctly through the system.\\

• \textbf{Simulated User Scenarios:} Develop tests that simulate real

user behaviors to check that the recommendation engine produces sensible

outputs.



\textbf{3. Regression Testing}\\

• \textbf{Continuous Integration:} Integrate regression tests into your

CI/CD pipeline to catch any unintended changes in recommendation

outcomes after modifications or rule adjustments.



\textbf{4. Performance and Load Testing}\\

• \textbf{Scalability Tests:} As the recommendation engine might be part

of a larger system, it's important to test how it performs under various

load conditions.\\

• \textbf{Response Time Benchmarks:} Establish benchmarks to ensure that

recommendations are generated within acceptable time limits, even as

data volume grows.



\begin{center}\rule{0.5\linewidth}{0.5pt}\end{center}



\section{Transitioning to Machine Learning

Later}\label{transitioning-to-machine-learning-later}



• \textbf{Extensibility:} By keeping the heuristic engine modular and

rule-driven, you create a solid baseline that can be used to compare

performance against future ML-based approaches.\\

• \textbf{Parallel Development:} Continue gathering data and user

feedback that can later serve as training data for a machine learning

model, ensuring a smoother transition when you're ready to integrate ML

methods.


\end{document}
